\documentclass[10pt,a4paper]{jsarticle}
\usepackage{bm}
\usepackage{graphicx}
\usepackage[truedimen,left=25truemm,right=25truemm,top=25truemm,bottom=25truemm]{geometry}
\usepackage{array}
\usepackage{titlesec}
\usepackage{jpdoc}

\titleformat*{\section}{\large\bfseries}
\def\title{秘密保持契約書}

\newcounter{NumOfMembers}\setcounter{NumOfMembers}{3}
\newcounter{DocumentsPerMember}\setcounter{DocumentsPerMember}{1}
\def\alias#1{\十干{#1}}
%\def\alias#1{\十干{#1}}
%\def\alias#1{\八卦{#1}}
%\def\alias#1{\十二支{#1}}
%\def\alias#1{\二十四山{#1}}	
%\def\alias#1{\干支{#1}}
%\def\alias#1{\八卦{#1}}
%\def\alias#1{\東西南北天地{#1}}
%\def\alias#1{\左右前後上下{#1}}
%\def\alias#1{\水金地火木土天海冥{#1}}
%\def\alias#1{\日月火水木金土{#1}}
%\def\alias#1{\イロハ{#1}}
%\def\alias#1{\いろは{#1}}
%\def\alias#1{\アイウ{#1}}
%\def\alias#1{\あいう{#1}}
%\def\alias#1{\alphabet{#1}}
%\def\alias#1{\Alphabet{#1}}
%\def\alias#1{\alphabet{#1}}
%\def\alias#1{\Alphabet{#1}}
%\def\alias#1{\numeric{#1}}
%\def\alias#1{\numeric{#1}}
%\def\alias#1{\漢数字{#1}}
%\def\alias#1{\丸数字{#1}}
%\def\alias#1{\Greece{#1}}
%\def\alias#1{\greece{#1}}
%\def\alias#1{\Cyrillic{#1}}
%\def\alias#1{\cyrillic{#1}}

\def\Name#1{\ifcase#1\or
株式会社甲種工業 \or%甲
株式会社乙種産業 \or%乙
兵庫県\or%己
\else \fi}



\def\Address#1{\ifcase#1\or
東京都千代田区丸の内一丁目1番1号 \or%甲
東京都千代田区千代田1番1号 \or%丙
兵庫県神戸市中央区下山手通五丁目10番1号\or%己
\else \fi}

\def\Representative#1{\ifcase#1\or
代表取締役 高橋一 \or%甲
代表取締役 田中太郎 \or%乙
兵庫県知事 乙野太郎\or%己
\else \fi}
\def\BlankSignature#1{\ifcase#1\or
0
\else
0\fi}

\newcommand\Target{甲乙間での協業可能性の検討及びその実施}
\newcommand\StartDate{本日}

\begin{document}
	\newpage
{\centering \Large\bf \title  \vskip 0em}
\vskip 2em
{\FirstSentence}は、\Target に関連し、当事者間で相互に開示される秘密情報の取扱いに関し、次の通り契約(以下「本契約」という)を締結する。

\article{目的}
本契約において秘密情報(以下「本秘密情報」という。)とは、文書・口頭その他有形無形を問わず、情報を開示する側(以下「情報開示者」という。)から、その開示された情報を受領する側(以下「情報受領者」という)に対し、開示される一切の情報のうち、情報開示者が秘密情報として明示的に特定したものをいう。ただし、次の各号に定めるものは、本秘密情報から除外する。
\begin{enumerate}
	\itm 情報開示者が開示した際に既に公知であった情報
	\itm 情報開示者が開示した後に情報受領者の責めによらないで公知となった情報
	\itm 情報開示者が開示した際に既に情報受領者が秘密保持義務を負うことなく保持していた情報
	\itm 情報受領者が秘密保持義務を負うことなく独自に第三者から入手した情報
	\itm 情報受領者が情報開示者から開示された情報によらずして独自に開発した情報
\end{enumerate}
\label{purpose}

\article{秘密情報保持}
情報受領者は、本秘密情報について厳に秘密を保持し、本件に関連して本秘密情報を必要とする情報受領者の役職員等(役員、従業員及び指揮監督下にある労働者派遣法のいう派遣労働者をいう。)並びに情報受領者が依頼する弁護士及び公認会計士(以下「本受領権者」という。)以外の者に対し本秘密情報を一切開示又は漏洩してはならず、本件に関連する以外の目的で本秘密情報を使用又は流用してはならない。
\term 情報受領者は、情報開示者の書面による承諾を得ることなく、本件に関連する以外の目的のために利用し、第三者に利用させ、又は開示若しくは漏洩をしてはならない。
\term 情報受領者が法令、規則、裁判所の決定・命令、行政庁の命令・指示等により本秘密情報の開示を要求された場合には、情報受領者は、情報開示者に対しその旨を直ちに通知することにより、情報開示者に、本秘密情報の開示・公開に反対するための手続きを行う機会を与えるものとする。この場合、情報開示者は、本秘密情報の機密性を確保するためにとりうる一切の措置を適切かつ迅速に行うことが出来るものとする。また、情報受領者がかかる開示を行う場合においても、法律上要求される必要最小限の内容、範囲と認められる部分についてのみ開示を行わなければならない。
\term 第1項に従い、情報受領者が本受領権者に本秘密情報の開示を行う場合には、情報受領者は、本秘密情報の機密性について本受領権者に対し十分かつ適切に説明し、本秘密情報について本契約による情報受領者の義務と同等の秘密保持義務(以下「本秘密保持義務」という)を負うことを確認するものとする。


\article{秘密情報の管理}
情報受領者は、本秘密情報(本秘密情報の記録された記録媒体、複写、複製又は翻訳物で、情報開示者に返還可能又は返還不能のいずれであるかを問わない。以下同じ。)を、本受領権者以外の者が接触、閲覧及びアクセスできないように、厳重に保管及び管理しなければならない。

\article{秘密情報の返還}
情報受領者は、本秘密情報のうち返還可能なものについては、情報開示者が要求したときは直ちに、情報開示者の指示に従い、原本およびその写しの一切を、情報開示者に返還しなければならない。
\term 情報受領者は、本秘密情報のうち返還不能なものについては、情報開示者が要求したときは直ちに、情報開示者の指示に従い、その一切を消去又は廃棄処分しなければならない。
\term 情報受領者は、前1項により本秘密情報を返還又は消去又は廃棄した後においても、かかる本秘密情報の内容に関し、本契約に基づく秘密保持義務を負う。
\term 本秘密情報に該当しない情報等については、情報受領者は情報開示者に対して前2項の返還、消去・廃棄処分、秘密保持の義務を負わない。

\article{関係者への遵守徹底}
情報受領者は、本秘密情報を知る事となる自己の役職員等に、本契約の内容を遵守させるものとする。

\article{非保証}
\AllMembers{}は、本契約の締結によっては、本件に関し、本契約に定めるもの以外に相互に何らの権利を取得し、又は義務を負うものではないことを相互に認める。

\article{有効期限}
本契約の有効期限は、\StartDate の日から1年間とする。ただし、\AllMembers{}が期間満了の1ヶ月前までに全ての相手方に対し書面による本契約を終了させる旨の通知を行わなかった場合には、本契約は1年間延長されるものとし、以後も同様とする。

\article{合意管轄}
本契約に関し、訴訟の必要が生じた場合には、東京地方裁判所を第一審の専属的合意管轄裁判所とする。

\article{協議}
本契約に定めのない事項又は疑義が生じた事項については、信義誠実の原則に従い当事者間で協議し、円満に解決を図るものとする。

\article{準拠法}
本契約は、抵触法の原則にかかわらず、日本法に準拠し、同法によって解釈されるものとする。

\article{言語}
本契約が、複数の言語で作成された場合について、同契約書の言語間の矛盾又は相違がある場合には、すべての点において日本語を優先するものとする。


\vspace{10pt}

以上の合意を証するために、{\LastSentence}


\begin{flushleft}
\today\\
\vspace{10pt}
\MakeSignatureField


\end{flushleft}
\end{document}
 
%


